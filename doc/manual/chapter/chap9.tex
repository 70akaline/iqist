\chapter{{\iqist} IN ACTION}
You may have learnt a lot about \iqist from the previous chapters. In this chapter, we will show you how to do calculations using \iqist through many case studies. Before this, we presume that you have successfully installed \iqist package in your systems, and the install directory is /opt/iqist, and the directory /opt/iqist/bin has been added to the PATH environment variable.

\noindent\colorbox{pink}{\parbox[r]{\linewidth}{\quad \$ export PATH=/opt/iqist/bin:\$PATH }}

If you are new to \iqist, please follow the tutorials to learn step by step. We believe that you will have become a master of \iqist after finishing all the tutorials. If you have been familiar with \iqist, there is no need to start from the beginning, you can choose some cases with your intesets to learn directly.

\section{Basic applications}
\subsection{Hello {\iqist}!}
We first use the \azalea component to solve the half-filled single band Hubbard model on the Bethe lattice self-consistently. Due to the density-density Coulomb interaction form in this model, we can use the \azalea component based on the segment picture. Let's start the journey!

\underline{Step 1: create a working directory, for example, iqsit\_test maybe a good choice.}

\noindent\colorbox{pink}{\parbox[r]{\linewidth}{\quad \$ mkdir iqist\_test }}

change directory into iqist\_test:

\noindent\colorbox{pink}{\parbox[r]{\linewidth}{\quad \$ cd iqist\_test }}

create a new directory t911:

\noindent\colorbox{pink}{\parbox[r]{\linewidth}{\quad \$ mkdir t911 }}

and change directory into t911:

\noindent\colorbox{pink}{\parbox[r]{\linewidth}{\quad \$ cd iqist\_test }}

now, the current directory is iqist\_test/t911.

\underline{step 2: prepare input file solver.ctqmc.in}

In the directory /opt/iqist/tutor/t911, we have prepared the input file solver.ctqmc.in for \azalea, please copy it to the current directory.

\noindent\colorbox{pink}{\parbox[r]{\linewidth}{\quad \$ cp /opt/iqist/tutor/t911/solver.ctqmc.in .}}

Here, we explain some key control parameters:
\begin{itemize}
  \item isscf = 2, self-consistent calculation
  \item issun = 2, symmetrize orbital
  \item isspn = 1, symmetrize spin
  \item isbin = 2, use data binning mode
  \item nband = 1, one band in this model
  \item nspin = 2, two projections of spin (up and down)
  \item norbs = 2, number of orbitals including spin
  \item ncfgs = 4, number of atomic configurations
  \item niter = 20, maximal of self-consistent iterations
  \item $U$ = $U_{c}$ = $U_{v}$ = 2.0, Coulomb interaction strength
  \item $J_{z}$ = $J_{s}$ = $J_{p}$ = 0.0, Hund's rule exchange strength
  \item mune = 1.0, chemical potential
  \item beta = 40.0, inversion of temperature ($\beta=\frac{1}{K_{B}T}$)
  \item part = 0.50, hopping parameters among nearest neighbours
  \item mkink = 1024, maximal diagram expansion order 
  \item ntime = 1024, number of points for imaginary-time mesh
  \item nflip = 2000, period for flipping spin
  \item nsweep = 20,000,000, total times of Monte Carlo sampling
  \item nwrite = 2,000,000, period for output
  \item nclean = 100,000, period for clean operation
  \item nmonte = ncarlo = 100, period for measurement of physical quantites
\end{itemize}

\underline{Step 3: run\azalea to do calculation}

In this case, we use 8 CPU cores to run azalea.x, you can adjust it according to your computer resources, and of course, you can also just use 1 CPU core to run azalea.x in sequential mode.

For example, use 8 CPU cores to run azalea.x in parallel mode:

\noindent\colorbox{pink}{\parbox[r]{\linewidth}{\quad \$ nohup mpiexec -n 8 azalea.x < /dev/null > out.dat \&}}

and use 1 CPU core to run azalea.x in sequential mode:

\noindent\colorbox{pink}{\parbox[r]{\linewidth}{\quad \$ nohup azalea.x < /dev/null > out.dat \&}}

\underline{Step 4: monitor the status of\azalea}

\underline{Step 5: analyze the calculation results}


\subsection{Mott metal-insulator transition}
The users can not only execute the components of the $i$QIST software package directly, but also invoke them in their own programs. To achieve this, we provide simple application programming interfaces (APIs) for most of the components in the $i$QIST software package in the Fortran, C, C++, and Python languages. With these well-defined APIs, one can easily setup, start, and stop the CT-HYB impurity solvers. For example, one can use the following Python script fragment to start the CT-HYB impurity solver:
\begin{verbatim}
    import mpi4py
    import pyiqist as iqist
    ...
    iqist.api.init_ctqmc(myid = 0, num_procs = 10)
    iqist.api.exec_ctqmc(iter = 20)
    iqist.api.stop_ctqmc()
\end{verbatim}
When the computations are finished, one can also collect and analyze the calculated results with Python scripts. Using these APIs, the users enjoy more freedom to design and implement very complex computational procedures and to adapt them to their own requirements.

\section{Advanced applications I: Complex systems}
\subsection{General Coulomb interaction}
The users can not only execute the components of the $i$QIST software package directly, but also invoke them in their own programs. To achieve this, we provide simple application programming interfaces (APIs) for most of the components in the $i$QIST software package in the Fortran, C, C++, and Python languages. With these well-defined APIs, one can easily setup, start, and stop the CT-HYB impurity solvers. For example, one can use the following Python script fragment to start the CT-HYB impurity solver:
\begin{verbatim}
    import mpi4py
    import pyiqist as iqist
    ...
    iqist.api.init_ctqmc(myid = 0, num_procs = 10)
    iqist.api.exec_ctqmc(iter = 20)
    iqist.api.stop_ctqmc()
\end{verbatim}
When the computations are finished, one can also collect and analyze the calculated results with Python scripts. Using these APIs, the users enjoy more freedom to design and implement very complex computational procedures and to adapt them to their own requirements.
\subsection{Spin-orbital coupling}
\subsection{Crystal field splitting}
\subsection{Retarded interaction and dynamical screening effect}

\section{Advanced applications II: Accurate measurement of physical observables}
\subsection{One-shot and self-consistent calculations}
The users can not only execute the components of the $i$QIST software package directly, but also invoke them in their own programs. To achieve this, we provide simple application programming interfaces (APIs) for most of the components in the $i$QIST software package in the Fortran, C, C++, and Python languages. With these well-defined APIs, one can easily setup, start, and stop the CT-HYB impurity solvers. For example, one can use the following Python script fragment to start the CT-HYB impurity solver:
\begin{verbatim}
    import mpi4py
    import pyiqist as iqist
    ...
    iqist.api.init_ctqmc(myid = 0, num_procs = 10)
    iqist.api.exec_ctqmc(iter = 20)
    iqist.api.stop_ctqmc()
\end{verbatim}
When the computations are finished, one can also collect and analyze the calculated results with Python scripts. Using these APIs, the users enjoy more freedom to design and implement very complex computational procedures and to adapt them to their own requirements.
\subsection{Data binning mode}
\subsection{Imaginary-time Green's function}
\subsection{Matsubara Green's function and self-energy function}
\subsection{Spin-spin correlation function and orbital-orbital correlation function}
\subsection{Two-particle Green's function and vertex function}

\section{Advanced applications III: post-processing procedures}
The users can not only execute the components of the $i$QIST software package directly, but also invoke them in their own programs. To achieve this, we provide simple application programming interfaces (APIs) for most of the components in the $i$QIST software package in the Fortran, C, C++, and Python languages. With these well-defined APIs, one can easily setup, start, and stop the CT-HYB impurity solvers. For example, one can use the following Python script fragment to start the CT-HYB impurity solver:
\begin{verbatim}
    import mpi4py
    import pyiqist as iqist
    ...
    iqist.api.init_ctqmc(myid = 0, num_procs = 10)
    iqist.api.exec_ctqmc(iter = 20)
    iqist.api.stop_ctqmc()
\end{verbatim}
When the computations are finished, one can also collect and analyze the calculated results with Python scripts. Using these APIs, the users enjoy more freedom to design and implement very complex computational procedures and to adapt them to their own requirements.
\subsection{Analytical continuation for imaginary-time Green's function}
\subsection{Analytical continuation for Matsubara self-energy function}

\section{Practical exercises}
\subsection{Orbital-selective Mott transition in two-band Hubbard model}
\subsection{Orbital Kondo and spin Kondo effects in three-band Anderson impurity model}
The users can not only execute the components of the $i$QIST software package directly, but also invoke them in their own programs. To achieve this, we provide simple application programming interfaces (APIs) for most of the components in the $i$QIST software package in the Fortran, C, C++, and Python languages. With these well-defined APIs, one can easily setup, start, and stop the CT-HYB impurity solvers. For example, one can use the following Python script fragment to start the CT-HYB impurity solver:
\begin{verbatim}
    import mpi4py
    import pyiqist as iqist
    ...
    iqist.api.init_ctqmc(myid = 0, num_procs = 10)
    iqist.api.exec_ctqmc(iter = 20)
    iqist.api.stop_ctqmc()
\end{verbatim}
When the computations are finished, one can also collect and analyze the calculated results with Python scripts. Using these APIs, the users enjoy more freedom to design and implement very complex computational procedures and to adapt them to their own requirements.

\section{Library mode}
\subsection{Call {\iqist} from Fortran language}
\subsection{Call {\iqist} from Python language}
