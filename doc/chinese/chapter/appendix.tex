% 附录
{
\appendix
\chapter{附录}
\renewcommand\thesection{A.\arabic{section}}
\renewcommand{\theequation}{A.\arabic{equation}}
\renewcommand{\thefigure}{A.\arabic{figure}}

\section{如何获得{\iqist}的最新版本?}
\label{app:get_code}

请发送电子邮件至huangli712@gmail.com或者是huangli712@yahoo.com.cn索取,请在邮件标题上注
明{\iqist}软件包字样并在邮件正文中留下您的工作单位以及详细的联系方式。经确认后,我们会
在1 $\sim$ 3个工作日内以附件的形式向您发送{\iqist}最新版本的安装包。

\section{如何获得本文档的最新版本?}
\label{app:get_doc}

请发送电子邮件至huangli712@gmail.com或者是huangli712@yahoo.com.cn索取,请在邮件标题上注
明{\iqist}用户手册字样并在邮件正文中留下您的工作单位以及详细的联系方式。经确认后,我们会
在1 $\sim$ 3个工作日内以附件的形式向您发送{\iqist}用户手册的最新版本。

\section{\iqist 的开发者是谁?}
\label{app:develop}

到目前为止(\today),{\iqist}的唯一开发者为本文作者黄理(四川材料与工艺研究所)。如果您对
{\iqist}的开发有兴趣,可以与我联系(mailto: huangli712@gmail.com或者huangli712@yahoo.com.cn)。

\section{\iqist 的维护者是谁?}
\label{app:maintain}

到目前为止(\today),{\iqist}的唯一维护者为本文作者黄理(四川材料与工艺研究所)。如果您对
{\iqist}的维护有兴趣,可以与我联系(mailto: huangli712@gmail.com或者huangli712@yahoo.com.cn)。

\section{如何获得技术支持?}
\label{app:support}

很遗憾,由于时间精力等方面的原因,对于用户在使用{\iqist}过程中所遇到的技术问题,我们目前
一般不提供无偿的技术支持。

\section{如何正确选取{\iqist}的组件?}
\label{app:choose}

在{\iqist}软件包中,提供了许多量子杂质求解器组件,那么究竟使用那一个组件呢?

首先请判断系统的相互作用类型,是否具有密度$-$密度相互作用形式亦或者为广义相互作用。
如果是前者,那么用户可以选用{\azalea}、{\gardenia}、{\narcissus}等组件;如果是后
者,那么可以选用{\begonia}、{\lavender}、{\camellia}、{\epiphyllum}、{\pansy}和
{\manjushaka}等组件。

其次请确定需要计算那些物理量,因为不同的量子杂质求解器组件,它能够测量的物理量是不
完全一样的。以双粒子格林函数为例,目前仅有{\gardenia}、{\narcissus}和{\lavender}组
件能够测量。至于动态屏蔽效应以及Hubbard-Holstein模型,目前仅有{\narcissus}组件能够
处理。关于量子杂质求解器组件的功能,请参阅第\ref{sec:function}节。

再次请依照效率优先原则,优先选用效率高的量子杂质求解器组件。例如,在基于段表示算法
的量子杂质求解器组件中,{\azalea}组件的效率最高,{\gardenia}组件次之,{\narcissus}
组件最差。如果没有特殊的要求,那么{\azalea}组件是最优的选择。

\section{什么时候可以使用正交多项式方法?}
\label{app:op}

当用户需要更为精确的测量结果,同时对计算速度的要求不是太高的时候,建议激活正交多
项式方法。详情请参阅第\ref{sec:isort}节、第\ref{subsec:itime}小节以及第\ref{subsec:ifreq}小节。

请注意,不是所有的量子杂质求解器组件均支持正交多项式功能,请参阅第
\ref{sec:function}节的描述。

\section{如何激活内核多项式功能?}
\label{app:kpm}

请打开ctqmc\_record.f90文件,找到cat\_make\_gtau()子程序,再找到
damp参数。damp参数用来指定不同的内核多项式。在缺省状态下,damp = 0,
这意味着不采用内核多项式方法。将damp参数置为1,这意味着采用Jackson
内核,经过我们的测试,该内核是最优的选择。当然了,用户也可以尝试其
它的选择。修改damp参数完毕后,请保存退出,重新编译程序产生ctqmc可
执行文件。此后用户的ctqmc程序将激活内核多项式功能。

请注意,不是所有的量子杂质求解器组件均支持内核多项式功能,请参阅第
\ref{sec:function}节的描述。

\section{如何固定占据数,搜索费米面?}
\label{app:fermi}

第一步:修改ctqmc\_main.f90文件,在call ctqmc\_dmft\_selfer()语句后面
加上对ctqmc\_dmft\_fermi()子程序的调用。

第二步:修改ctqmc\_dmft.f90文件,添加ctqmc\_dmft\_fermi()子程序。

第三步:ctqmc\_dmft\_fermi()子程序的核心算法如下:
\begin{itemize}
\item 定义目标占据数$N_{\text{final}}$

\item 计算当前占据数$N_{\text{curr}}$

\item 获取当前的化学势$\mu_{\text{curr}}$

\item 计算新的化学势$\mu_{\text{new}} = \mu_{\text{curr}} + (N_{\text{final}} - N_{\text{curr}})$

\item 如有必要,更新相关的物理量
\end{itemize}

第四步:保存所有修改,重新编译程序。

此算法的精度不是很高,实际占据数与目标占据数
之间的偏差大约在$\pm 0.01$左右。并且如果体系属于绝缘相,那么可能需要迭代计算很多
次才能得到所需要的结果(niter $\sim$ 60)。

\section{如何计算高阶关联函数?}
\label{app:corr}

请使用{\gardenia}组件或者是{\narcissus}组件,并查阅第\ref{sec:isvrt}小节关于isvrt
参数的叙述。

\section{如何考虑不同的晶格态密度?}
\label{app:dos}

第一步:产生合适的晶格态密度。用户既可以使用{\hibiscus}/hibiscus-toolbox组件中
的makedos程序来产生晶格态密度,亦可以自行构造晶格态密度。晶格态密度数据必须存储
在solver.anydos.in文件中,关于solver.anydos.in文件的详细格式,请参阅第\ref{sec:sai}
节。

第二步:修改ctqmc\_dmft.f90文件中的ctqmc\_dmft\_selfer()子程序,注释掉对
ctqmc\_dmft\_bethe()子程序的调用,改为对ctqmc\_dmft\_anydos()子程序的调用即可。

第三步:保存修改后的ctqmc\_dmft.f90文件,重新进行编译产生ctqmc程序。

第四步:正常进行DMFT自洽计算。

\section{出现segment fault错误时应该怎么办?}
\label{app:seg}

出现此错误多半是因为ctqmc程序所使用的堆栈空间超出了操作系统的限制。解决办法是
修改make.sys文件,在LEVEL选项后面添加-heap-arrays 8192,然后重新编译,重新运行
程序即可。如果问题还没有解决,请将相关的资料(包括程序源代码、输入文件、用户的
系统配置、编译环境配置和当前的随机数种子等等)发送给程序开发者。

\section{如何使用gfortran编译\iqist 组件程序?}
\label{app:gfortran}

使用gfortran编译\iqist 组件程序很简单,只需要修改make.sys文件中的以下几点:
\begin{itemize}
  \item 将预处理选项FPP=-fpp改为FPP=-cpp
  \item 注释掉机器优化选项CHECK、CDUMP、MTUNE的所有内容
  \item LEVEL选项只保留-O3选项,其余的全部注释掉
  \item LAPACK和BLAS也必须使用gfortran编译的版本
\end{itemize}

保存修改后的make.sys文件,然后重新编译即可。
}
