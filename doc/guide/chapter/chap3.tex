\chapter{{\iqist}的运行}
\label{chap:run}

本章讲述如何运行{\iqist}的量子杂质求解器组件。

\section{搭建{\iqist}运行平台}
\label{sec:platform}

{\iqist}可以在多种软硬件平台上执行。经过测试,{\iqist}至少可以在下述硬件平台上正常运行:
\begin{itemize}
\item HP 6730b laptop + Ubuntu Linux 10.04
\item MacPro Workstation + Mac OS X 10.6
\item Linux Cluster + Cent OS 5.0
\item DAWN4000 System
\item SHENTENG7000 System
\item TIANHE-1 System
\end{itemize}
上述硬件平台从笔记本到工作站,从小集群到大型机,涵盖的范围十分广,并且软件平台也是多种多
样,由此可见{\iqist}的移植性适用性非常好。

在编译安装成功{\iqist}后,无须特别的设置,即可执行它的各种组件。但是
在运行{\iqist}的量子杂质求解器组件之前,请确保以下条件均具备:

1.确保Intel Fortran Compiler的运行时刻库在系统路径中:

\noindent\colorbox{pink}{\parbox[r]{\linewidth}{\quad \$ source /opt/intel/Compiler/11.1/072/bin/ifortvars.sh intel64}}

如果用户没有使用Intel Fortran Compiler,而是使用gfortran或者是PGI Fortran Compiler,那么这一步可以略过。

2.确保MPI实现的可执行程序在系统路径中:

\noindent\colorbox{pink}{\parbox[r]{\linewidth}{\quad \$ export PATH=/usr/local/mpich/bin:\$PATH}}

如果用户打算在串行模式下执行量子杂质模型求解器,那么可以略过此步骤。

3.确保MPI环境的守护进程已经在后台运行:

\noindent\colorbox{pink}{\parbox[r]{\linewidth}{\quad \$ mpd \&}}

如果用户打算在串行模式下执行量子杂质模型求解器,那么可以略过此步骤。此外,由于实现机制不一样,某
些MPI实现并不要求启动守护进程,此时也可以略过此步骤。具体情况请参阅MPI实现的相关文档。

4.确保系统中的SSH服务已经打开。如果用户打算在串行模式下执行量子杂质模型求解器,那么可以略过此步骤。

\section{创建{\iqist}输入文件}
\label{sec:create_input}

\begin{table}
\centering
\begin{minipage}{\linewidth}
\caption{量子蒙特卡洛杂质求解器输入文件需求表\label{tab:input}}
\begin{tabular}{lccccc}
\hline
\hline
 & {\azalea} & {\gardenia} & {\narcissus} & {\begonia} & {\lavender} \\
\hline
solver.ctqmc.in  & {\color{green}Y}\footnote{{\color{green}Y}表示该文件是可选的。} 
& {\color{green}Y} & {\color{green}Y} & {\color{green}Y} & {\color{green}Y} \\
solver.eimp.in   & {\color{green}Y} & {\color{green}Y} & {\color{green}Y} & {\color{green}Y} & {\color{green}Y} \\
solver.hyb.in    & {\color{green}Y} & {\color{green}Y} & {\color{green}Y} & {\color{green}Y} & {\color{green}Y} \\
solver.anydos.in & {\color{green}Y} & {\color{green}Y} & {\color{green}Y} & {\color{green}Y} & {\color{green}Y} \\
solver.ktau.in   & N\footnote{N表示该文件将被忽略。} & N & {\color{green}Y} & N & N \\
atom.cix         & N & N & N & {\color{red}Y}\footnote{{\color{red}Y}表示该文件是必须的。} & {\color{red}Y} \\
\hline
\hline
\end{tabular}
\end{minipage}
\end{table}

{\iqist}的量子杂质模型求解器组件需要如下几种输入文件:
\begin{itemize}
\item solver.ctqmc.in, 包含所有控制参数,参见第\ref{sec:sci}节
\item solver.eimp.in,包含杂质能级$E_{\text{imp}}$以及对称矩阵,参见第\ref{sec:sei}节
\item solver.hyb.in,包含初始的杂化函数$\Delta(i\omega)$,参见第\ref{sec:shi}节
\item solver.anydos.in,包含晶格模型的态密度$N(\epsilon)$,参见第\ref{sec:sai}节
\item solver.ktau.in,包含双积分推迟作用函数$\mathcal{K}(\tau)$,参见第\ref{sec:ski}节
\item atom.cix,包含原子组态的信息,参见第\ref{sec:ac}节
\end{itemize}

那么很显然,对于不同的量子杂质模型求解器,所需要的输入文件是完全不同的。具体的对应情况请参见
表\ref{tab:input}。

对于solver.ctqmc.in文件与solver.eimp.in文件,你可以利用vim程序或者其它任意字处理软件手工编辑
标准输入文件,使其符合实际需求,也可以利用pysci.py程序辅助产生全新的solver.ctqmc.in文件与
solver.eimp.in文件\footnote{关于pysci.py程序的细节,请参阅第\ref{sec:pysci}节。}。

solver.hyb.in文件通常由上次的输出结果(solver.hyb.dat)转换而来,或者是由其它程序产生,无法
手工编辑。

solver.anydos.in文件可以由{\hibiscus}组件的toolbox/makedos程序产生,也可以由其它程序产生,
无法手工编辑\footnote{关于makedos程序的细节,请参阅第\ref{sec:hib-tool}节。}。

solver.ktau.in文件可以由{\hibiscus}组件的toolbox/makescr程序产生,无法手工编辑\footnote{关于makescr
程序的细节,请参阅第\ref{sec:hib-tool}节。}。

atom.cix文件可以由{\jasmine}组件产生,也可以由杜亮博士提供的原子问题程序rambutan产生,不能
手工编辑\footnote{关于{\jasmine}组件的细节,请参阅第\ref{sec:jasmine}节。关于rambutan程序
的问题,请与杜亮博士联系(mailto:duliang@iphy.ac.cn)。}。对于{\begonia}与{\lavender}组件而
言,atom.cix文件是必须的。除了这个特例以外,其它任意输入文件对于任意的量子杂质模型求解器而
言都是可选的输入,可以不提供。

\section{{\iqist}组件的运行}
\label{sec:run_iqist}

以串行模式运行量子杂质模型求解器的命令为:

\noindent\colorbox{pink}{\parbox[r]{\linewidth}{\quad \$ nohup ctqmc < /dev/null > out.dat \&}}

以并行模式运行量子杂质模型求解器的命令为:

\noindent\colorbox{pink}{\parbox[r]{\linewidth}{\quad \$ nohup mpiexec -n np ctqmc < /dev/null > out.dat \&}}

其中nohup程序让量子杂质模型求解器不会被挂起,忽略系统发送的hangup信号;mpiexec表示MPI启动程序,np表示计算
核心的数目,建议最好保持2的倍数;</dev/null 表示从空设备中读取输入文件;> out.dat表示将输出重定向至out.dat
文件中;\&符号让量子杂质模型求解器在后台执行。

如果你所用的服务器需要使用队列系统提交计算任务,那么上述命令须做出相应的改动。

\section{{\iqist}组件的监测}
\label{sec:monitor_iqist}

保持对{\iqist}组件,尤其是量子杂质求解器组件的运行时刻监测是十分重要的。可以通过下述方法监测{\iqist}组件的
运行情况。

1.使用top命令监测{\iqist}组件的进程信息,包括CPU占用率,内存占用,运行时间等等。在正常情况下,每个ctqmc进
程的CPU占用率应该在100\%左右。

2.在当前工作目录中执行下述命令:

\noindent\colorbox{pink}{\parbox[r]{\linewidth}{\quad \$ tail -f out.dat}}

随时注意{\iqist}组件的运行输出有无异常。每隔一段时间,量子杂质求解器组件会自检自身的运行状态,如果发现异常会
报错退出。

3.在编译{\iqist}组件时启用-pg选项,那么{\iqist}组件在运行时会自动搜集各个子程序的执行信息,包括调用次数,执
行时间等等。接着你可以在任意时刻强制终止{\iqist}组件的执行\footnote{通过CTRL+C或者是CTRL+D命令。},那么在当前
工作目录下会自动产生一个名为gmon.out的文件。然后利用gprof程序可以探测执行的瓶颈在何处。此前我们曾经频繁使用
此方法来优化{\iqist}组件的执行效率。

\noindent\colorbox{pink}{\parbox[r]{\linewidth}{\quad \$ gprof -b ctqmc}}

4.随时观察solver.green.dat与solver.hist.dat等输出文件,看当前的计算结果是否合理。

5.随时观察solver.status.dat文件,看计算结果是否合理,普通用户不建议使用此方法,因为我知道你看不懂。
