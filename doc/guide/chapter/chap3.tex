\chapter{RUNNING}
\section{Configure your system}
\section{Create input files}
\section{Execute codes}

To use $i$QIST is easy. At first, since there are several CT-HYB impurity solvers in the package and their features and efficiencies are somewhat different, it is the user's responsibility to choose suitable CT-HYB components to deal with the impurity problem at hand. Second, the $i$QIST is in essence a computational engine, so users have to write scripts or programs to execute the selected CT-HYB impurity solver directly or to call it using the application programming interface. For example, if the users want to conduct CT-HYB/DMFT calculations, they must implement the DMFT self-consistent equation by themselves. Third, an important task is to prepare proper input data for the selected CT-HYB impurity solver. The optional input for the CT-HYB impurity solver is the hybridization function [$\Delta(\tau)$ or $\Delta(i\omega_n)$], impurity level ($E_{\alpha\beta}$), interaction parameters ($U$, $J$, and $\mu$), etc. If users do not provide them to the impurity solver, it will use the default settings automatically. Specifically, if the Coulomb interaction matrix is general, users should use the \texttt{JASMINE} component to solve the local atomic problem at first to generate the necessary eigenvalues and eigenvectors. Fourth, execute the CT-HYB impurity solver. Finally, when the calculations are finished, users can use the tools contained in the \texttt{HIBISCUS} component to post-process the output data, such as the imaginary-time Green's function $G(\tau)$, Matsubara self-energy function $\Sigma(i\omega_n)$, and other physical observables. For more details, please refer to the user guide of $i$QIST.

\section{Monitor and Profile}
