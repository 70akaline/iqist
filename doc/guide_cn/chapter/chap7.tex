\chapter{{\iqist}附属工具程序}
\label{chap:tools}

除了量子杂质模型求解器以外,{\iqist}软件包中还包含了许多工具程序,可以完成
众多的前处理与后处理工作。本章将对这些工具程序的用法做全面的介绍。

\section{{\jasmine}组件}
\label{sec:jasmine}

{\jasmine}组件属于原子问题程序,它的作用在于对角化局域哈密顿量,得到相应的本
征值与本征态,产生$\mathcal{F}$矩阵,为广义相互作用版本的连续时间量子蒙特卡洛
杂质求解器组件提供atom.cix输入文件。{\jasmine}组件有三种运行模式,分别对应于
以下三类量子杂质求解器组件:

\begin{itemize}
\item {\begonia}、{\lavender}
\item {\camellia}、{\epiphyllum}
\item {\pansy}、{\manjushaka}
\end{itemize}

这三类量子杂质求解器组件的主体框架是类似的,区别在于计算trace部分所使用的算法
与技巧。{\begonia}与{\lavender}组件应用了稀疏矩阵技术与分而治之的策略。
{\camellia}与{\epiphyllum}组件应用了Krylov子空间迭代算法以及Newton-Leja多
项式插值算法。{\pansy}与{\manjushaka}组件应用了好量子数技术。这些量子杂质求解
器组件所需要的输入数据是大不相同的,自然地,相应的atom.cix文件的格式也是不大
相同的。

{\jasmine}组件的可执行程序名称为atom,所需要的输入文件为atom.in。典型的atom.in文
件如下所示,这是一个4带模型的配置。

\begin{lstlisting}[frame=single]
--------------------------------------------------------------------------
>>>     atom.in: configuration file for atomic eigenvalues problem     <<<
--------------------------------------------------------------------------
2        ! imode: running mode, 1 = eigen basis mode; 2 = occupation number basis mode
1        ! ifock: source of fock space, 1 = internal mode; 2 = external mode
-1       ! isoce: with or without SOC, -1 = without; 0 = s; 1 = p; 2 = d; 3 = f; 99 = debug mode
--------------------------------------------------------------------------
4        ! nband: number of correlated bands
2        ! nspin: number of spin projection
8        ! norbs: number of correlated orbitals (= nband * nspin)
256      ! ncfgs: number of atomic states (= 2**norbs)
--------------------------------------------------------------------------
6.00     ! U    : average Coulomb interaction
6.00     ! Uc   : intraorbital Coulomb interaction
3.00     ! Uv   : interorbital Coulomb interaction, Uv = Uc - 2 * Jz for t2g system
1.50     ! Jz   : Hund's exchange interaction in z axis (Jz = Js = Jp = J)
1.50     ! Js   : spin-flip term
1.50     ! Jp   : pair-hopping term
0.00     ! lsoc : spin orbital coupling strength
--------------------------------------------------------------------------
-9.75000 ! eimp(01) orbital, impurity level
-9.75000 ! eimp(02) orbital
-9.75000 ! eimp(03) orbital
-9.75000 ! eimp(04) orbital
-9.75000 ! eimp(05) orbital
-9.75000 ! eimp(06) orbital
-9.75000 ! eimp(07) orbital
-9.75000 ! eimp(08) orbital
 0.00000 ! eimp(09) orbital
 0.00000 ! eimp(10) orbital
 0.00000 ! eimp(11) orbital
 0.00000 ! eimp(12) orbital
 0.00000 ! eimp(13) orbital
 0.00000 ! eimp(14) orbital
\end{lstlisting}

各输入参数的简明解释如下,至于详细权威的解释请参阅{\jasmine}/atom\_control.f90
文件中的注释。由于我们还需要对{\jasmine}组件进行深度的重构,因此这些参数的名称
以及具体含义可能会发生较大的变化,用户需要特别注意。

\begin{itemize}
\item {\color{red}imode}

设定{\jasmine}的运行模式。

imode = 1时,所产生的atom.cix文件适合于{\begonia}与{\lavender}组件。

imode = 2时,所产生的atom.cix文件适合于{\camellia}与{\epiphyllum}组件。

imode = 3时,所产生的atom.cix文件适合于{\pansy}与{\manjushaka}组件,此选项暂不可用。

\item {\color{red}ifock}

设定Fock态的来源。

ifock = 1时,由{\jasmine}自身产生Fock态。

ifock = 2时,从外部文件fock.in中读取Fock态,此选项暂时不可用。

\item {\color{red}isoce}

是否考虑自旋$-$轨道耦合项(SOC)。

isoce = -1时,不支持SOC。

isoce = 0时,考虑$l = 0$,$s$轨道的情况。

isoce = 1时,考虑$l = 1$,$p$轨道的情况。

isoce = 2时,考虑$l = 2$,$d$轨道的情况。

isoce = 3时,考虑$l = 3$,$f$轨道的情况。

isoce = 99时,进入debug模式。

由于当前版本的{\jasmine}组件不支持SOC,因此isoce必须设为-1,否则所产生的atom.cix
文件是不正确的。

\item {\color{red}nband}

能带的数目。

此参数雷同于solver.ctqmc.in文件中的nband参数,请参阅第\ref{sec:nband}节。

\item {\color{red}nspin}

自旋取向的数目,恒定为2。

此参数雷同于solver.ctqmc.in文件中的nspin参数,请参阅
第\ref{sec:nspin}节。

\item {\color{red}norbs}

轨道的数目,等于nband $ \times $ nspin。

此参数雷同于solver.ctqmc.in文件中的norbs参数,请参阅第\ref{sec:norbs}节。

\item {\color{red}ncfgs}

原子组态的数目,等于$2^{\text{norbs}}$。

此参数雷同于solver.ctqmc.in文件中的ncfgs参数,请参阅第\ref{sec:ncfgs}节。

\item {\color{red}$U$}

平均Coulomb相互作用。

此参数在程序中实际上不起作用,通常设为$U = U_{c}$。此参数雷同
于solver.ctqmc.in文件中的$U$参数,请参阅第\ref{sec:U}节。

\item {\color{red}$U_{c}$}

轨道内的Coulomb相互作用。

此参数雷同于solver.ctqmc.in文件中的$U_{c}$参数,
满足$U_{c} = U_{v} - 2*J_{z}$关系,请参阅第\ref{sec:Uc}节。

\item {\color{red}$U_{v}$}

轨道间的Coulomb相互作用。

此参数雷同于solver.ctqmc.in文件中的$U_{v}$参数,
满足$U_{c} = U_{v} - 2*J_{z}$关系,请参阅第\ref{sec:Uv}节。

\item {\color{red}$J_{z}$}

Hund交换常数。

此参数雷同于solver.ctqmc.in文件中的$J_{z}$参数,满
足$U_{c} = U_{v} - 2*J_{z}$关系,请参阅第\ref{sec:Jz}节。

\item {\color{red}$J_{s}$}

自旋翻转项(spin-flip)常数,通常设为$J_{z} = J_{s} = J_{p}$。

此参数雷同于solver.ctqmc.in文件中的$J_{s}$参数,请参阅第\ref{sec:Js}节。

\item {\color{red}$J_{p}$}

对跃迁项(pair-hopping)常数,通常设为$J_{z} = J_{s} = J_{p}$。

此参数雷同于solver.ctqmc.in文件中的$J_{p}$参数,请参阅第\ref{sec:Jp}节。

\item {\color{red}lsoc}

自旋$-$耦合作用强度。

此参数仅当isoce不等于-1时起作用。

\item {\color{red}eimp}

杂质轨道能级$E_{\text{imp}}(i)$。

此参数实际上是一个一维数组,先安排自旋朝上的轨道,后安排自旋朝下的轨道,没有用到
的轨道全部都置为0。

\end{itemize}

{\jasmine}组件在运行后,会输出以下文件:

\begin{itemize}
\item atom.cix,最重要的输出文件,量子杂质求解器组件就需要它,文件格式请参阅第\ref{sec:ac}节。
\item atom.eimp.dat,包含杂质轨道能级信息。
\item atom.eval.dat,包含原子问题的本征值。
\item atom.evec.dat,包含原子问题的本征态。
\item atom.fmat.dat,包含$\mathcal{F}$矩阵的数据。
\item atom.fock.dat,包含Fock态的信息。
\item atom.hist.dat,统计每个本征态的简并度。
\item atom.hmat.dat,包含局域哈密顿量的非零矩阵元。
\item atom.smat.dat,包括自旋$-$轨道耦合矩阵。
\item atom.umat.dat,包括Coulomb相互作用矩阵的非零矩阵元。
\end{itemize}

除了{\jasmine}组件以外,利用特别定制的rambutan程序也能生成合格的atom.cix文件,计
算结果是一致的。如果系统中包含了自旋$-$轨道耦合作用,那么rambutan程序是唯一的选
择。关于rambutan程序的详情请与杜亮博士联系。

\section{{\hibiscus}/hibiscus-entropy组件}
\label{sec:hib-ent}

{\hibiscus}组件包含了四个子组件,分别为:
\begin{itemize}
\item entropy
\item stochastic
\item swing
\item toolbox
\end{itemize}
其中前三个子组件主要瞄准观测量(如虚时格林函数与虚频电子自能函数)的解析延拓过程,最
后的toolbox子组件主要用于前处理与后处理任务。从本节开始,我们将依次详细介绍
上述四个子组件。

hibiscus-entropy组件的作用是对虚时格林函数$G(\tau)$进行解析延拓,获得电子谱函
数$A(\omega)$。hibiscus-entropy组件的实现基于最大熵方法,这是解析延拓领域的经
典算法,应用十分广泛\cite{jarrell:133}。

hibiscus-entropy组件的可执行程序为entropy,所需要的主要输入文件为entropy.in。
entropy.in文件包含了hibiscus-entropy组件的全部控制参数,典型的entropy.in文件
如下所示:

\begin{lstlisting}[frame=single]
--------------------------------------------------------------------------
>>> entropy.in: config parameters for maximum entropy method code
--------------------------------------------------------------------------
129    ! ntime : number of imaginary time slice
200    ! nwmax : number of frequency point on half axis
20     ! niter : number of cycles for maximum entropy method
16     ! ntune : number of smooth runs for maximum entropy method
3000   ! nstep : number of annealing steps per maximum entropy method cycle
1      ! nband : number of bands
2      ! norbs : number of orbitals
1      ! ntype : type of default model, if ntype = 0, gaussion type, if ntype = 1, flat type
--------------------------------------------------------------------------
1200.  ! ainit : initial alpha parameter
0.001  ! devia : it is the deviation from the green's function
8.000  ! beta  : inversion of real temperature
1.600  ! sigma : gauss broadening parameter
0.025  ! wstep : frequency step, used to build the frequency mesh
^^^^^^^^^^^^^^^^^^^^^^^^^^^^^^^^^^^^^^^^^^^^^^^^^^^^^^^^^^^^^^^^^^^^^^^^^^
\end{lstlisting}

各输入参数的简明解释如下,至于详细权威的解释请参阅hibiscus-entropy/entropy\_control.f90
文件中的注释。

\begin{itemize}
\item {\color{red}ntime}

虚时间轴$[0,\beta)$上所划分的时间片段的数目。

请注意,如果量子杂质求解器采用的是{\daisy}组件,那么此处的ntime应该等
于solver.hfqmc.in文件中的ntime的值加1。如果量子杂质求解器基于CT-HYB算
法,那么此处的ntime就直接等于solver.ctqmc.in文件中的ntime。关于量子杂
质求解器组件的ntime参数的详细信息,请参阅第\ref{sec:ntime}节。

\item {\color{red}nwmax}

定义实频率点的数目。

在hibiscus-entropy组件中,默认实频数据点是围绕0点对称分布的。亦即在正
半轴有nwmax个数据点,在负半轴也有nwmax个数据点,一共2*nwmax + 1个数据
点。数据点之间的间隔为wstep,总共覆盖的频率区间为[-nwmax*wstep,+nwmax*wstep]。

\item {\color{red}niter}

最大熵方法的最大迭代计算次数。

\item {\color{red}ntune}

谱函数$A(\omega)$微调次数。

最大熵方法迭代计算完毕后,还需对获得的谱函数进行细微调整以及平滑处理,ntune
表示进行后期调整的次数。

\item {\color{red}nstep}

Monte Carlo sampling次数。

在最大熵算法的每次迭代计算中,所需要进行的退火步数,可以理解为Monte Carlo
抽样次数。

\item {\color{red}nband}

能带的数目。

此参数雷同于solver.ctqmc.in文件中的nband参数。关于量子杂
质求解器组件的nband参数的详细信息,请参阅第\ref{sec:nband}节。

\item {\color{red}norbs}

轨道的数目,norbs = nband $ \times $ 2。

此参数雷同于solver.ctqmc.in文件中的norbs参数。
关于量子杂质求解器组件的norbs参数的详细信息,请参阅第\ref{sec:norbs}节。

\item {\color{red}ntype}

指定缺省模型$m(\omega)$的类型。

ntype = 0时,采用Gauss模型;

ntype = 1时,采用常数模型。

\item {\color{red}ainit}

初始的$\alpha$参数。

ainit参数通常设得比较大,一般大于1000。

\item {\color{red}devia}

$G(\tau)$的数值偏差程度。

该参数用来描述虚时格林函数数据的涨落,取0.001即可,通常无须改动。

\item {\color{red}beta}

反温度$\beta = 1/T$。

此参数雷同于solver.ctqmc.in文件中的beta参数。
关于量子杂质求解器组件的beta参数的详细信息,请参阅第\ref{sec:beta}节。

\item {\color{red}sigma}

Gauss模型中的展宽参数。

该参数仅当ntype = 0时起作用。

\item {\color{red}wstep}

实轴频率数据点之间的间隔。

该参数也就是俗称的步长,由wstep参数与nwmax参数可以定义整个实轴网格。

\end{itemize}

除了entropy.in文件外,hibiscus-entropy组件还需要的输入文件为tau.grn.dat。该文件包
含了虚时格林函数$G(\tau)$的数据,可由hibiscus-toolbox组件的maketau程序产生(请参阅
第\ref{sec:hib-tool}节)。

hibiscus-entropy组件的输出文件比较少,仅有mem.dos.dat
与mem.sum.dat,前者包含谱函数$A(\omega)$的数据,后者包含了$A(\omega)$的sum rules数
据,普通用户不必关心。用户还可以利用hibiscus-toolbox组件中的makekra程序以及makeups
程序对mem.dos.dat文件进行进一步的处理,计算出更多的物理量,详情请参阅第\ref{sec:hib-tool}节。

\section{{\hibiscus}/hibiscus-stochastic组件}
\label{sec:hib-sai}

hibiscus-stochastic组件的作用是对虚时格林函数$G(\tau)$进行解析延拓,获得电子
谱函数$A(\omega)$。也就是说hibiscus-stochastic组件的作用与hibiscus-entropy组
件的作用是相同的,只不过前者基于随机解析延拓方法\cite{beach},而后者基于最大
熵方法\cite{jarrell:133}。从原理上说最大熵方法仅仅是随机解析延拓方法的一个特
例,随机解析延拓方法更为先进,解析延拓的结果更为可靠。但是随机解析延拓方法的
计算效率很低,需要长时间的计算才能得到比较平滑的计算结果,因此在日常的应用中
我们建议用户还是选择hibiscus-entropy组件进行解析延拓,而采用hibiscus-stochastic
组件的计算结果作为对照。

hibiscus-stochastic组件的可执行文件为sai,它所需要的主要输入文件为sai.in。sai.in
包含了全部的控制参数,典型的sai.in文件示例如下:

\begin{lstlisting}[frame=single]
--------------------------------------------------------------------------
>>> sai.in: config parameters for stochastic analytic continuation code
--------------------------------------------------------------------------
1024       ! ntime : number of imaginary time slice
400        ! nwmax : number of frequency point on half axis
10001      ! ngrid : number of slice of x in [0,1]
1024       ! ngamm : number of r_{\gamma} and a_{\gamma}
1          ! nalph : number of alpha parameter used in parallel tempering
2000       ! nwarm : maximum number of thermalization steps
200000     ! nstep : maximum number of quantum Monte Carlo sampling steps
2000       ! ndump : output period for stochastic analytic continuation code
2          ! ltype : measurement scheme
64         ! lemax : maximum number of legendre polynomial
20001      ! legrd : number of mesh points for legendre polynomial
--------------------------------------------------------------------------
1.000      ! ainit : initial alpha parameter
2.000      ! ratio : \alpha_(p+1) / \alpha_p = R
8.000      ! beta  : inversion of real temperature
0.005      ! eta1  : lorentz broadening parameter \eta_1
0.000      ! sigma : gauss broadening parameter
0.020      ! wstep : frequency step, used to build the real frequency mesh
^^^^^^^^^^^^^^^^^^^^^^^^^^^^^^^^^^^^^^^^^^^^^^^^^^^^^^^^^^^^^^^^^^^^^^^^^^
\end{lstlisting}

那么sai.in文件中各个输入参数的简明解释如下。至于详细权威的解释请参
阅hibiscus-stochastic/sai\_control.f90
文件中的注释。

\begin{itemize}
\item {\color{red}ntime}

虚时间轴$[0,\beta)$上所划分的时间片段的数目。

虚时格林函数$G(\tau)$定义在$[0,\beta)$上,这一区间被划分为ntime个时间片,以便
于处理。该参数须等于solver.ctqmc.in文件中设置的ntime。
关于量子杂质求解器组件的ntime参数的详细信息,请参阅第\ref{sec:ntime}节。

\item {\color{red}nwmax}

定义实频率点的数目。

在hibiscus-stochastic组件中,默认实频数据点是围绕0点对称分布的。亦即在正半
轴有nwmax个数据点,在负半轴也有nwmax个数据点,一共2*nwmax + 1个数据点。数据
点之间的间隔为wstep,总共覆盖的频率区间为[-nwmax*wstep,+nwmax*wstep]。

\item {\color{red}ngrid}

用于定义稠密的[0, 1]区间内的线性网格。

hibiscus-stochastic组件内部使用一个平滑映射$\phi$将谱函数$A(\omega)$存在的
频率区间$[−\infty, +\infty]$
映射至[0, 1]区间:
\begin{equation}
\phi(\omega) = x
\end{equation}
并且$x \in [0,1]$。再将[0, 1]区间均分为ngrid个间隔。经过简单的数学变换后,
谱函数$A(\omega)$可以用无量纲场$n(x)$表示出来。随机解析延拓方法实质上是使
用Monte Carlo方法抽样求解$\langle n(x) \rangle$,
然后在程序的最后才将其转换为$A(\omega)$。

\item {\color{red}ngamm}

$\delta$函数的数目。

在随机解析延拓算法中,使用大量$\delta$函数的叠加来刻画无量纲场$n(x)$:
\begin{equation}
\label{eq:ncx}
n_{\mathcal{C}}(x) = \sum_{\gamma} r_{\gamma} \delta(x - a_{\gamma}).
\end{equation}
$r_{\gamma}$为$\delta$函数的权重,满足$r_{\gamma} > 0 $,而且
$\sum_{\gamma}r_{\gamma} = 1$。$a_{\gamma}$为$\delta$函数的位置,满足
$0 \leq a_{\gamma} \leq 1$。$\gamma$为$\delta$函数的索引,一般$2^{10}$个$\delta$
函数即可取得令人满意的结果。$\mathcal{C} = \{r_{\gamma}, a_{\gamma}\}$表示某时刻
无量纲场的构型,一旦$\mathcal{C}$确定下来,那么$n(x)$也就确定了。我们使用经典蒙特
卡洛方法来计算$\langle n(x) \rangle$,应用Metropolis算法,随机选择$\mathcal{C}$中
任意一个$\delta$函数,尝试修改其位置与权重:$a_{\gamma} \mapsto a^{\prime}_{\gamma}$
以及$r_{\gamma} \mapsto r^{\prime}_{\gamma}$,场构型也会发生相应的变化:$\mathcal{C} 
\mapsto \mathcal{C}^{\prime}$。ngamm参数即为$\delta$函数的数目,从原则上说,ngamm
越大,那么计算精度越高,但是计算效率越低。

\item {\color{red}nalph}

$\alpha$参数的数目。

如果nalph大于1,那么parallel tempering算法自动启用。

\item {\color{red}nwarm}

Monte Carlo抽样算法的热平衡步数。

此参数雷同于solver.ctqmc.in文件中的ntherm参数。
关于量子杂质求解器组件的ntherm参数的详细信息,请参阅第\ref{sec:ntherm}节。

\item {\color{red}nstep}

Metropolis算法的Monte Carlo抽样次数。

此参数雷同于solver.ctqmc.in文件中的nsweep参数。
关于量子杂质求解器组件的nsweep参数的详细信息,请参阅第\ref{sec:nsweep}节。

\item {\color{red}ndump}

Monte Carlo抽样算法的输出间隔。

亦即每间隔ndump次抽样,hibiscus-stochastic组件会输出一次当前的计算结果。此
参数雷同于solver.ctqmc.in文件中的nwrite参数。
关于量子杂质求解器组件的nwrite参数的详细信息,请参阅第\ref{sec:nwrite}节。

\item {\color{red}ltype}

谱函数的测量方式。

ltype = 1时,采用标准测量方式;

ltype = 2时,那么采用Legendre正交多项式算法改善测量精度。

\item {\color{red}lemax}

Legendre正交多项式的展开阶数。

此参数仅当ltype = 2时有意义。此参数雷同于solver.ctqmc.in文件中的lemax参数。
关于量子杂质求解器组件的lemax参数的详细信息,请参阅第\ref{sec:lemax}节。

\item {\color{red}legrd}

用于定义稠密的[-1, 1]区间内的线性网格。

在[-1,1]间建立一个线性网格,Legendre正交多项式即定义在此线性网格上,legrd
为网格点的数目。此参数雷同于solver.ctqmc.in文件中的legrd参数。
关于量子杂质求解器组件的legrd参数的详细信息,请参阅第\ref{sec:legrd}节。

\item {\color{red}ainit}

第一个$\alpha$参数的值。

\item {\color{red}ratio}

前后两个$\alpha$参数的比值$R$,仅当nalph参数大于1时才有意义。

\begin{equation}
\frac{\alpha_{i+1}}{\alpha_{i}} = R
\end{equation}

当hibiscus-stochastic组件启动时,即会根据nalph、ainit与ratio的值创
建$\alpha$参数的列表$\alpha_{i}$。

\item {\color{red}beta}

反温度$\beta = 1/T$。

此参数雷同于solver.ctqmc.in文件中的beta参数。
关于量子杂质求解器组件的beta参数的详细信息,请参阅第\ref{sec:beta}节。

\item {\color{red}eta1}

$\eta$参数,用于定义$\delta(x)$函数:
\begin{equation}
\delta(x) = \frac{\eta}{x^{2} + \eta^{2}}
\end{equation}

\item {\color{red}sigma}

Gauss展宽参数$\sigma$,用于设定缺省模型$m(\omega)$。

如果$\sigma > 0$,那么采用Gauss模型为缺省模型,
\begin{equation}
m(\omega) = \exp{\left(-\frac{\omega^{2}}{2\sigma^{2}}\right)}
\end{equation}
如果$\sigma \leq 0$,那么采用常数模型为缺省模型:
\begin{equation}
m(\omega) = \text{constants}
\end{equation}

\item {\color{red}wstep}

实轴频率数据点之间的间隔。

该参数也就是俗称的步长,由wstep参数与nwmax参数可以定义整个实轴网格。

\end{itemize}

除了sai.in文件外,hibiscus-stochastic组件还需要的输入文件为tau.grn.dat。
该文件包含了虚时格林函数$G(\tau)$的数据,可由hibiscus-toolbox组件的
maketau程序产生(请参阅第\ref{sec:hib-tool}节)。值得注意的是
hibiscus-stochastic组件每次仅能处理单个轨道的数据。因此如果用户面临的
是多带系统,那么需要将tau.grn.dat文件按轨道拆分为多个同名文件,然后
用hibiscus-stochastic组件依次进行处理。这也是hibiscus-stochastic组件
不如hibiscus-entropy组件的地方之一。

hibiscus-stochastic组件运行时刻输出的文件比较多,具体如下所示。

\begin{itemize}
\item sai.image.dat,存储谱函数$A_{\alpha}(\omega)$。每个$\alpha$参数都对应着一个谱函
数,因此sai.image.dat文件中有很多个block,每个block前有$\alpha$参数的数值,然后是
$A_{\alpha}(\omega)$,block与block之间间隔两个空行。
\item sai.imsum.dat,存储谱函数$A(\omega)$,它实际上是$A_{\alpha}(\omega)$的算术平均值。这
就是我们所需要求的谱函数。用户可将sai.imsum.dat文件重命名为mem.dos.dat,然后使用hibiscus-toolbox
组件中的makekra和makeups程序对其进行后处理,详情可参阅第\ref{sec:hib-tool}节。
\item sai.move.dat,存储MOVE UPDATE的Monte Carlo抽样统计信息,也就是该操作的接受概率。
\item sai.swap.dat,存储SWAP UPDATE的Monte Carlo抽样统计信息,也就是该操作的接受概率。
\item sai.ppleg.dat,存储Legendre正交多项式展开系数,仅当ltype = 2时才有意义,普通用
户对此无须关心。
\end{itemize}

\section{{\hibiscus}/hibiscus-swing组件}
\label{sec:hib-swing}

hibiscus-swing组件的作用是对虚频电子自能函数$\Sigma(i\omega)$进行解析延拓,得到实频
电子自能函数$\Sigma(\omega)$。hibiscus-swing组件来源于K. Haule开发的解析延拓程序\cite{haule:195107},
后经我们重新封装改进。它的主要原理是使用Gaussian多项式拟合$\Sigma(i\omega)$,关于具体
的细节,请参考原始文献。

hibiscus-swing组件的主体是使用Python语言开发,部分计算模块采用Fortran 90语言编写。在
编译安装hibiscus-swing组件之前,请确保系统中已经正确安装了f2py。hibiscus-swing组件所
需要的Python运行环境为2.6+,但是暂不支持3.0+。如要正常执行hibiscus-swing组件,还需要
scipy、numpy等Python数值库的支持。

在应用hibiscus-swing组件之前,请先预判系统是处于金属态还是绝缘态。hibiscus-swing组件
内含两个启动脚本,metal.sh与insulator.sh,分别对应上述两种情况。换言之,如果系统属于
金属态,那么执行脚本metal.sh以启动hibiscus-swing的主程序,反之则使用insulator.sh脚本。
metal.sh脚本的示例如下:
\begin{lstlisting}[frame=single]
exec=../../../../hibiscus-swing/swing_main.py
para='-sig std.sgm.dat -nom 128 -beta 10.0 -wexp 1.15 -Ng 60 -FL True ...'
python $exec $para
\end{lstlisting}
insulator.sh脚本的示例如下:
\begin{lstlisting}[frame=single]
exec=../../../../hibiscus-swing/swing_main.py
para='-sig std.sgm.dat -nom 128 -beta 40.0 -wexp 1.05 -Ng 120 -FL False ...'
python $exec $para
\end{lstlisting}
metal.sh与insulator.sh脚本的解释如下:

第1行是设定hibiscus-swing组件的主程序所在的位置。主程序名称为swing\_main.py。

第2行设定hibiscus-swing组件的启动参数,-sig选项用来指定包含虚频电子自能函数
的文件,-nom选项用来指定一共考虑多少个松原频率点,-beta选项用来指定反温度,
-FL选项用来指定Fermi liquid行为是否需要保持。至于其余选项的具体含义,请参阅
swing\_main.py文件,此处不再赘述。

第3行的作用是启动hibiscus-swing组件的主程序。

hibiscus-swing组件的输入文件通常是std.sgm.dat,该文件由hibiscus-toolbox组件
中的makestd程序产生,详情请参阅第\ref{sec:hib-tool}节。需要特别提醒的是:目
前hibiscus-swing组件每次只能处理一个轨道的数据。对于多带系统而言,如果轨道
之间是简并的,那么只运行一次hibiscus-swing组件即可。如果轨道之间不是简并的,
那么需要对std.sgm.dat文件进行预先拆分,拆分后的文件仍然命名为std.sgm.dat,
只不过每个std.sgm.dat文件中包含的自能数据都互不相同而已。接下来需要用
hibiscus-swing组件依次处理所有的std.sgm.dat文件。我们建议用户针对不同的
轨道(亦即不同的std.sgm.dat文件)建立不同的文件夹,各自进行处理,以免产生
混淆。

hibiscus-swing组件的输出文件比较多,主要有如下几类:

\begin{itemize}
\item gaus.*,*代表迭代计算的次数,下同。gaus.*文件主要存放与Gaussian多项
式展开系数有关的数据,普通用户对此不必关心。
\item siom.*,此文件主要存放(利用当前的Gaussian多项式展开系数计算得到的)
虚频电子自能函数$\Sigma_{\text{fit}}(i\omega)$。用户可以比较原始自能数据
$\Sigma(i\omega)$与此拟合数据之间的差异。
\item sres.*,此文件主要存放迭代计算过程中所获得的实频电子自能函
数$\Sigma_{\text{iter}}(\omega)$。普通用户不必关心。
\item sigr\_linear.out,此文件为最终的计算结果,存放$\Sigma_{\text{fit}}(\omega)$。
注意:sigr\_linear.out文件中仅仅包含一个轨道的电子自能数据。如果用户面对的
是多带系统,那么除了拆分std.sgm.dat文件之外,还有一件麻烦事要做,那就是把
把若干个sigr\_linear.out文件合并起来。
\end{itemize}

hibiscus-swing组件在运行过程中不需要用户去干预。通常需要$O(10)$量级的迭代
次数才能收敛,如果虚频电子自能函数的数据质量比较差(换言之,曲线不够平滑),
那么程序将很难收敛。

除了hibiscus-swing组件以外,利用hibiscus-toolbox组件中的makesig程序也能对
虚频电子自能函数$\Sigma(i\omega)$进行解析延拓,获得$\Sigma(\omega)$。只不过
makesig程序利用了P\'{a}de展开方法,在技术路线上与swing稍有不同,详情请参阅
第\ref{sec:hib-tool}节。

\section{{\hibiscus}/hibiscus-toolbox组件}
\label{sec:hib-tool}

hibiscus-toolbox组件内包含了许多小工具程序,它们的主要作用是为量子杂质求解器组件
提供前处理与后处理,下面将分门别类地介绍它们的功能与具体用法。

\subsection{makechi}
makechi程序的作用是由自旋$-$自旋关联函数计算有效局域磁矩$M_{e}$。具体计算公式如下:
\begin{equation}
M_e = \sqrt{T \chi_{loc}},
\end{equation} 
其中磁化系数$\chi_{loc}$的定义式为:
\begin{equation}
\label{eq:chi_eq}
\chi_{loc} = \int^{\beta}_{0} \text{d}\tau \chi_{loc}(\tau) 
           = \int^{\beta}_{0} \text{d}\tau \langle S_{z}(0) S_{z}(\tau) \rangle,
\end{equation}
其中$\chi_{loc}(\tau) = \langle S_z(0) S_z(\tau) \rangle$就是自旋$-$自旋关联函
数的表达式。

makechi程序需要的输入文件为solver.schi.dat,无输出文件。
在执行makechi程序时,用户只需按照屏幕输出的提示一步一步操作即可。在程序结束时,屏幕上
会输出$\chi_{loc}$与$M_{e}$的数值。

使用{\gardenia}与{\narcissus}量子杂质求解器组件可以计算自旋$-$自旋关联函数,相应
的控制参数为isvrt = 2,详情请参阅第\ref{sec:isvrt}节。

\subsection{makedos}
makedos程序的作用是计算几种典型的晶格模型的电子态密度。目前最新版本的makedos程序支
持以下几种情况:

\begin{itemize}
\item 无穷维立方晶格,Gaussian型态密度, 输出文件:dos.gauss.dat
\item 简立方晶格,输出文件:dos.cubic.dat
\item bethe晶格,半圆态密度:dos.bethe.dat
\item Lorentzian型态密度,输出文件:dos.loren.dat
\end{itemize}

makedos程序无须任何输入文件,对于不同的态密度,输出文件的名称也不尽相同,具
体如上所示。在执行makedos程序时,用户只需按照屏幕输出的提示一步一步操作即可。
用户可将dos.*.dat文件重命名为solver.anydos.in文件,然后提供给量子杂质求解器
组件作为输入。关于solver.anydos.in文件的详情,用户可以参阅第\ref{sec:sai}节。

\subsection{makekra}
makekra程序的作用是由电子谱函数$A(\omega)$计算出实频杂质格林函数的虚部
$\Im G(\omega)$,
\begin{equation}
\label{eq:a_img}
A(\omega) = -\frac{1}{\pi}\Im G(\omega).
\end{equation}
再利用Kramers-Kr\"{o}nig变换关系:
\begin{equation}
\Re G(\omega) = -\frac{1}{\pi} \int^{\infty}_{-\infty} \text{d}\omega^{\prime} 
\frac{\Im G (\omega)}{\omega - \omega^{\prime}},
\end{equation}
进一步求出实部$\Re G(\omega)$。

目前makekra程序与hibiscus-entropy程序相互接口,它需要的输入文件为mem.dos.dat,
输出文件为kra.grn.dat。在执行makekra程序时,用户只需按照屏幕输出的提示一
步一步操作即可。

\subsection{makescr}
makescr程序的作用是由屏蔽谱函数$W(\omega)$计算出双积分推迟作用函数$\mathcal{K}(\tau)$,
以及相互作用强度$U$与化学势$\mu$的调整量。

$\mathcal{K}(\tau)$的定义如下\cite{werner:2010}:
\begin{equation}
\label{eq:k_tau}
\mathcal{K}(\tau) = \int^{\infty}_{0} \frac{d\omega'}{\pi} \frac{\text{Im} W(\omega')}{\omega'^{2}}
\left[\mathcal{B}(\omega',\tau) - \mathcal{B}(\omega',0)\right].
\end{equation}

相互作用强度$U$的调整量由下式计算,调整量即为下式右端第二项:
\begin{equation}
U_{scr} = U + 2\int^{\infty}_{0} \frac{\text{d}\omega^{\prime}}{\pi} \frac{\Im{W(\omega^{\prime})}}{\omega^{\prime}}.
\end{equation}

化学势$\mu$的调整量由下式计算,调整量即为下式右端第二项:
\begin{equation}
\mu_{scr} = \mu + \int^{\infty}_{0} \frac{\text{d}\omega^{\prime}}{\pi} \frac{\Im {W(\omega^{\prime})}}{\omega^{\prime}},
\end{equation}

在执行makescr程序时,用户只需按照屏幕输出的提示一步一步操作即可。
makescr程序所需要的输入文件为scr.frq.dat\footnote{scr.frq.dat文
件中的数据通常由文献中获取。},其中包含了屏蔽谱函数$W(\omega)$的
原始数据。输出文件为scr.tau.dat,其中包含了$\mathcal{K}(\tau)$的
数据。makescr程序的实际作用是为{\narcissus}组件准备输入文件。在
通常情况下,我们需要将scr.tau.dat重命名为solver.ktau.in(请参阅
第\ref{sec:ski}节),以便与{\narcissus}组件配合使用。相关的控制参
数为isvrt = 99,详情请参阅第\ref{sec:isvrt}节。

\subsection{makesig}
makesig程序的作用是对量子杂质求解器组件输出的虚频电子自能函数
$\Sigma(i\omega)$数据进行解析延拓,得到实频上的电子自能函数
$\Sigma(\omega)$。此程序十分简单,所采用的算法为P\'{a}de算
法。如果原始电子自能函数的数据质量不佳,那么此程序给出的结果
不一定是合理的,用户需要自行甄别。

makesig程序需要的输入文件为solver.sgm.dat,输出文件为sig.sgm.dat。
在执行makesig程序时,用户只需按照屏幕输出的提示一步一步操作即可。

hibiscus-swing组件的功能也是对虚频电子自能函数进行解析延拓,这两
个程序的计算结果可用于相互比较,详情请参阅第\ref{sec:hib-swing}节。

\subsection{makestd}
makestd程序的作用是对量子杂质求解器组件输出的虚频电子自能函
数$\Sigma(i\omega)$数据进行预处理,为自能解析延拓程
序(亦即hibiscus-swing组件,详情请参阅第\ref{sec:hib-swing}节)
提供合适的输入文件。

makestd程序需要的输入文件为solver.sgm.dat.*,*表示自1起的连续
数字,没有上限,和solver.green.bin.*的情况类似\footnote{很遗憾,
目前{\iqist}并没有提供一个类似于data binning的机制,自动生成
solver.sgm.dat.*。因此,为了得到一系列的solver.sgm.dat.*文件,用
户必须重复多次以单步模式执行量子杂质求解器组件,并手动重命名
所获得的solver.sgm.dat文件。}。makestd程序的
输出文件为std.sgm.dat。在执行makestd程序时,用户只需按照屏幕输
出的提示一步一步操作即可。

\subsection{maketau}
maketau程序的作用是对量子杂质求解器组件输出的虚时格林函数$G(\tau)$
数据进行预处理,为解析延拓程序(hibiscus-entropy组件与
hibiscus-stochastic组件,请参阅第\ref{sec:hib-ent}节与
第\ref{sec:hib-sai}节)提供合适的输入文件。maketau程序支持以下四种情况。

\begin{itemize}
\item CT-QMC组件(除{\daisy}组件外的其余量子杂质求解器组件,下同)输出的solver.green.dat文件
\item HF-QMC组件(仅{\daisy}组件,下同)输出的solver.green.dat文件
\item CT-QMC组件输出的solver.green.bin.*文件
\item HF-QMC组件输出的solver.green.bin.*文件
\end{itemize}

其中前两种情况的处理结果适合hibiscus-entropy组件,而后两种情况的处理结果适合hibiscus-stochastic组件。
maketau程序所需的输入文件为solver.green.dat或者是solver.green.bin.*,输出文件为tau.grn.dat。
在执行maketau程序时,用户只需按照屏幕输出的提示一步一步操作即可。

\subsection{makeups}
makeups程序的作用是由谱函数$A(\omega)$计算紫外光电子能谱以及X射线吸收谱。

紫外光电子能谱的计算:杂质谱函数$A(\omega)$乘以Fermi-Dirac分布函数$f(\omega)$
即可获得理论紫外光电子能谱,如下式所示:
\begin{equation}
A_{\text{PES}}(\omega) = A(\omega) f_{T}(\omega) 
= \frac{A(\omega)}{ 1 + e^{\frac{\omega - \mu}{k_{B}T}}}.
\end{equation}
由于实验仪器的分辨率是有限的,为了能够与实验数据相比较,须得考虑对理论曲线进
行Gaussian展宽,具体计算公式如下所示:
\begin{equation}
\label{eq:gaussian_broading}
\tilde{A}_{\text{PES}}(\omega) = \int \text{d}\omega^{\prime} 
\frac{1}{\sqrt{2\pi}\sigma} 
e^{\frac{(\omega - \omega^{\prime})^{2}}{2\sigma^{2}}} 
A_{\text{PES}}(\omega^{\prime}),
\end{equation}
其中$\sigma$为展宽因子,对于高分辨率的紫外光电子能谱实验,$\sigma$可以取为0.1,
至于其它情况$\sigma$的取值范围在$0.2\sim 0.36$之间,大致上与实验分辨率保持一致。

X射线吸收谱的计算:杂质谱函数$A(\omega)$乘以反Fermi-Dirac分布函数$f(-\omega)$
即可获得理论X射线吸收谱,如下式所示:
\begin{equation}
A_{\text{XAS}}(\omega) = \frac{A(\omega)}{ 1 + e^{\frac{\mu - \omega}{k_{B}T}}}.
\end{equation}
为了能够与实验谱线相比较,同样须得对上述理论曲线进行Gaussian展宽操作,具体计算
公式与公式(\ref{eq:gaussian_broading})完全一致,展宽参数$\sigma$的取值范围
在$0.2 \sim 0.4$之间。

makeups程序的输入文件为mem.dos.dat,该文件由hibiscus-entropy程序产生,内含谱函
数$A(\omega)$的数据,详情请参阅第\ref{sec:hib-ent}节。输出文件为ups.pes.dat与
ups.xas.dat,分别包含紫外光电子能谱数据$A_{\text{PES}}(\omega)$以及X射线吸收谱
数据$A_{\text{XAS}}(\omega)$。在执行makeups程序时,用户只需按照屏幕输出的提示
一步一步操作即可。

\section{pysci.py程序}
\label{sec:pysci}

{\iqist}的量子杂质求解器组件对于输入文件的格式有十分严格的要求,一个额外的
空行或者是字符都有可能造成严重的错误(请参阅第\ref{sec:sci}节)。为了解决这个
问题,在{\iqist}软件包中提供了一个名为pysci.py的Python脚本,利用此脚本,我
们可以快速地产生标准的solver.ctqmc.in输入文件。

如要运行pysci.py脚本,系统中首先需要安装Python 3.0+。

第一步:

在执行pysci.py之后,脚本会提示用户选择量子杂质求解器组件。目前pysci.py脚本仅仅支持
{\azalea}、{\gardenia}和{\narcissus}组件。用户只需要输入量子杂质求解器组件名称的首
字母(大小写均可),并按下回车键,即可选定一种组件。如果用户不提供任何输入,只是直接
按下回车键,那么脚本默认用户选择了{\azalea}组件。

第二步:

根据用户选定的组件名称,脚本程序会自动产生一系列的参数。对于每个参数,脚本程序都会
显示参数的内部名称、数据类型、缺省值以及详细具体的解释。用户需要输入合适的数值,并
按下回车键。然后脚本程序会提示下一个参数。这样直至所有必须的参数都被设定完毕。如果
用户不输入任何数值,只是简单的按下回车键,那么脚本程序将把缺省值赋给当前的参数。此
外,脚本程序对于用户所输入数据的合理性以及有效性不做任何检测。我们将决定权都留给用
户。

第三步:

根据用户的输入,脚本程序将产生标准的solver.ctqmc.in文件,输出到当前目录下。pysci.py
程序只保证solver.ctqmc.in文件格式的准确性,而不保证文件内容的合理性。用户需要根据自
身的需求,仔细检查solver.ctqmc.in文件中的设定,并做出相应的调整。

在后续版本的pysci.py程序中,我们将添加更多的组件支持,并引入用户输入数据检测功能。
